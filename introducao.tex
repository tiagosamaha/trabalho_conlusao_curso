\chapter{INTRODUÇÃO}

Para lidar com a dinâmica da economia atual, empresas de software estão sofrendo fortes pressões para desenvolver sistemas de informação em prazos cada vez menores, uma vez que existem muitos clientes não satisfeitos com a qualidade eu tempo de entrega dos sistemas. Tais sistemas precisam ser escaláveis e integrados com outros sistemas existentes ou em desenvolvimento. Os ambientes tecnológicos nos quais estes sistemas são desenvolvidos estão em constante evolução devido a solicitações do cliente \cite{CRESPO}.

As metodologias tradicionais de desenvolvimento de software já não se adequam às demandas atuais por agilidade na mudança e flexibilidade na sua construção, em especial no caso dos softwares voltados para gestão organizacional. Em um projeto tradicional, os programadores recebem uma documentação inicial, que descreve o que o software deve fazer, e precisam transformar o texto escrito em código.

Com as constantes modificações dos requisitos do sistema que visam acompanhar os processo de negócio, a sincronização da documentação torna-se um grave problema. Documentos escritos em papel, ou até mesmo digitalizados, precisam de pessoas para mantê-los atualizados e estão mais sujeitos a erros, por faltar sincronia com o código fonte, que, em última instância, é o que realmente faz o software funcionar. Pior que ter documentação em excesso é ter documentação errada. É preferível ter pouca documentação a ter uma grande quantidade de documentação com erros, pois estes podem levar a mal-entendidos que resultarão em prejuízos - tais como dias de trabalho perdido \cite{NSI}.

Um outro grande problema no processo de desenvolvimento de software é conseguir garantir que todos os requisitos solicitados pelo cliente realmente estão funcionando como esperado, assim surge a necessidade de criar testes automatizados que cubram todos os requisitos solicitados. Testes automatizados são programas ou scripts simples que monitoram funcionalidades do sistema sendo testado e fazem verificações automáticas nos resultados obtidos. A grande vantagem desta abordagem, é que todos os casos de teste podem ser facilmente e rapidamente repetidos a qualquer momento e com pouco esforço. Além disso, testes automatizados são uma boa documentação para o desenvolvedor \cite{KON}.

A solução destes problemas é a utilização de testes automatizados – código que verifica código – associados a chamada documentação executável. Entende-se por documentação executável, a definição de cenários expressos de execução do sistema de informação pelo cliente em linguagem natural, que permitem definição e manutenção de requisitos que irão validar e verificar o software desenvolvido. Assim o cliente terá uma documentação que irá garantir a qualidade do software, além de ajudar no treinamento de utilização do mesmo.

\section{Justificativa do trabalho}

Através deste trabalho desejamos criar um material que se torne referência, uma vez que não se encontra na literatura amplo material sobre o assunto. O desenvolvimento de uma ferramenta de documentação executável foi o tema escolhido para este trabalho através do contato de um parceiro industrial (Nexedi SA) do Núcleo de Pesquisa em Sistemas de Informação (NSI), que desenvolve pesquisas em sistemas integrados de gestão.

\section{Objetivo}

O objetivo deste projeto é integrar a aplicações web (portais, sistemas integrados de gestão), uma ferramenta que permite ao usuário final executar cenários – tutorial guiado – objetivando a validação dos requisitos do sistema. Desta forma, é possível simultaneamente validar os requisitos do sistema e guiar o usuário no emprego do sistema.