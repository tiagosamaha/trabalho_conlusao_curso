\chapter{INTRODUÇÃO}

Para lidar com a dinâmica da economia atual, empresas de software estão sofrendo fortes pressões para desenvolver sistemas de informação em prazos cada vez menores, uma vez que existem muitos clientes não satisfeitos com a qualidade e o tempo de entrega destes. Tais sistemas precisam ser escaláveis e integrados com outros sistemas existentes ou em desenvolvimento. Adicionalmente, os ambientes tecnológicos nos quais estes sistemas são desenvolvidos estão em constante evolução devido a solicitações do cliente \cite{CRESPO}.

As metodologias tradicionais de desenvolvimento de software já não se adequam às demandas atuais por agilidade na mudança e flexibilidade na sua construção, em especial no caso dos softwares voltados para gestão organizacional. Em um projeto tradicional, os programadores recebem uma documentação inicial, que descreve o que o software deve fazer, e precisam transformar o texto escrito em código.

Com as constantes modificações dos requisitos do sistema visando acompanhar os processo de negócio, a sincronização da documentação torna-se um grave problema. Documentos escritos em papel, ou até mesmo digitalizados, precisam de pessoas para mantê-los atualizados e estão mais sujeitos a erros, por faltar sincronia com o código fonte, que, em última instância, é o que realmente faz o software funcionar. Pior que ter documentação em excesso é ter documentação errada. É preferível ter pouca documentação a ter uma grande quantidade de documentação com erros, pois estes podem levar a mal-entendidos que resultarão em prejuízos - tais como dias de trabalho perdidos \cite{NSI}.

Um outro grande problema no processo de desenvolvimento de software é conseguir garantir que todos os requisitos solicitados pelo cliente realmente estejam funcionando como esperado, assim, surge a necessidade de testes de software, isso é, a verificação do software confrontando cada requisito documentado com a implementação do mesmo.

Uma solução para este procedimento é o uso de testes automatizados. São programas ou scripts simples que monitoram as funcionalidades do sistema sendo testado, fazendo verificações automáticas nos resultados obtidos. A grande vantagem desta abordagem, é que todos os casos de teste podem ser facilmente e rapidamente repetidos a qualquer momento e com pouco esforço. Além disso, testes automatizados são uma boa documentação para o desenvolvedor \cite{KON}.

Entretanto, o uso de testes automatizados – código que verifica código – não resolve sozinho os problemas de sincronia entre código e documentação, por isso utiliza-se testes automatizados associados à chamada documentação executável. Entende-se por documentação executável neste trabalho, a definição de cenários de execução do sistema de informação expressos pelo cliente em linguagem natural, que permitem definição e manutenção de requisitos que irão validar e verificar o software desenvolvido.

Segundo \citeonline{PRESSMAN}, sistemas cuja documentação esteja desatualizada são considerados legados. Como neste caso a documentação é executável e depende do próprio sistema, esta ferramenta mitiga os riscos de perda de sincronismo entre documentação e seu código, minimizando os problemas adivindos de sistemas legados.

Atráves da utilização destas técnicas no processo de desenvolvimento, o cliente terá uma documentação sempre atualizada, além de garantir a qualidade do software e ajudar no treinamento de utilização do mesmo.

\section{Objetivo}

O objetivo deste projeto é integrar a aplicações \textit{web} a uma ferramenta que permite ao usuário final executar cenários – tutorial guiado – objetivando a validação dos requisitos do sistema. Desta forma, é possível simultaneamente validar os requisitos do sistema e guiar o usuário no emprego do sistema.

\section{Justificativa do trabalho}

Agregar ao projeto da Biblioteca Digital os valores de treinamento, documentação e validação do sistema, resultantes do uso desta ferramenta. Além disto, deseja-se criar um material que se torne referência, uma vez que encontra-se pouca referência na literatura sobre este tema.

\section{Estrutura do trabalho}

No Capítulo 2, é feita uma fundamentação de conceitos e técnicas necessárias para o entendimento deste trabalho. No Capítulo 3 é descrito o estudo de caso e as etapas de desenvolvimento da ferramenta NSI-BD-Tour. Por fim, no Capítulo 4 apresenta-se as conclusões, limitações e os trabalhos futuros a serem desenvolvidos.