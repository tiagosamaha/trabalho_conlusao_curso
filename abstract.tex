\begin{center}
\textbf{ABSTRACT}
\end{center}
\singlespacing

\noindent The aim of this conclusion work is to present the executable documentation, applied in the Digital Library project, as an evolution of traditional documentation, bringing advantages such as possibility of user training and functionality validation. Additionally, this new documentation paradigm also tends to stay synchronized, because these documents also assumes other functionalities for the user. As a result, this study contributed to Digital Library project (SETEC/EPCT), and have described the use of Amberjack2 and jQuery libraries. \\

\noindent KEYWORDS: executable documentation, automated tests, training