\chapter{CONSIDERAÇÕES FINAIS}

\section{Conclusões}

Com o trabalho realizado foi possível prototipar e avaliar uma ferramenta que possibilite ao usuário descrever os requisitos do sistema em forma de cenários. Essa conquista viabilizou a execução destes cenários em forma de tutorial guiado, trazendo um melhor entendimento do sistema, e tornando mais fácil e conexa a verificação de que todos os requisitos tenham sido corretamente implementados.

O conceito que envolve a prototipação da ferramenta, levou a uma comparação entre documentação tradicional e executável. Em um projeto tradicional, as constantes modifições dos requisitos do sistema geram problemas de sincronização dos documentos escritos em papel, tornando os sistemas legados. Já a documentação executável pôde ser descrita em linguagem aproximada a natural, semelhante à tradicional, com a vantagem de poder ser executada, além de validar os cenários documentados automaticamente.

Espera-se com este trabalho ter demonstrado que o conceito da ferramenta de documentação executável pode ser utilizado por outros projetos, no intuito de oferecer ao cliente uma documentação que agregue mais valor ao seu sistema. Com isso os usuários poderão ser mais facilmente treinados, e terão garantias da validação dos módulos do sistema.

\section{Limitações}

A ferramenta NSI-BD-Tour, atualmente, consegue tratar apenas um subconjunto da interações entre o usuário e o sistema.

Esta ferramenta, precisa ser melhor testada, desenvolvendo um número maior de cenários, objetivando verificar sua adequação para a validação das funcionalidades de sistemas maiores que o sistema da Biblioteca Digital.

Outras estruturas mais complexas, como por exemplo laços e decisões, precisam ser previstas na linguagem de definição de cenário.

\section{Trabalhos Futuros}

Como trabalho futuro, pode-se melhorar a linguagem de descrição dos cenários. Apesar da forma como esta funcionalidade está atualmente implementada, utilizando o formato \textit{YAML}, facilitar o gerenciamento dos cenários, ainda é uma forma pouco natural e propensa a erros de sintaxe e de identação.

Esta ferramenta poderá ainda, ser desacoplada do \textit{framework Ruby on Rails}, possibilitando sua utilização em outros ambientes de programação.

O tratamento de erros durante a execução dos cenários prevê a produção de um relatório de inconformidades identificadas, entretanto esta funcionalidade ainda não foi implementada.

Estudar o conceito de ontologias para a definição de cenários e fluxos de trabalho a serem executadas pela ferramenta.