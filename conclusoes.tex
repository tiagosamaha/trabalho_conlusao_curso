\chapter{CONSIDERAÇÕES FINAIS}

\section{Conclusões}

Com o trabalho realizado foi possível prototipar e avaliar uma ferramenta que possibilite ao usuário descrever os requisitos do sistema em forma de cenários. Essa conquista possibilitará a execução destes cenários em forma de tutorial guiado, trazendo um melhor entendimento do sistema, e tornando mais fácil e conexa a verificação de que todos os requisitos foram corretamente implementados.

O conceito que envolve a prototipação da ferramenta, levou a discussão da documentação tradicional e executável. Em um projeto tradicional, as constantes modifições dos requisitos do sistema geram problemas de sincronização dos documentos escritos em papel. Já a documentação executável pôde ser descrita em linguagem natural, da mesma forma que a tradicional, porém pôde ser executada, e trouxe a vantagem da possibilidade de validar os cenários documentados automaticamente.

Espera-se com este trabalho ter demonstrado que o conceito da ferramenta de documentação executável pode ser utilizado por outros projetos, no intuito de oferecer ao cliente uma documentação que agregue mais valor ao seu sistema. Com isso os usuários poderão ser mais facilmente treinados, e ajudarão na validação dos módulos do sistema.

\section{Limitações}

\section{Trabalhos Futuros}