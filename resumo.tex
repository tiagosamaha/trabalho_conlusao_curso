\begin{center}
\textbf{RESUMO}
\end{center}
\singlespacing

\noindent O objetivo deste trabalho é apresentar o conceito de documentação executável, aplicada no projeto da Biblioteca Digital como uma evolução da documentação tradicional, trazendo como vantagens adicionais a possibilidade do treinamento dos usuários e a validação das funcionalidades. Além disso, este formato de documentação também tende a manter-se atualizado mais facilmente, já que além de documentar, a mesma assume outras funcionalidades, agregando mais valor ao usuário. Assim, como resultados diretos, este trabalho contribui com o projeto da Biblioteca Digital da SETEC/EPCT e descreve o emprego das bibliotecas \textit{Amberjack2} e jQuery na implementação de documentacao executável. \\

\noindent PALAVRAS-CHAVE: documentação executável, testes automatizados, treinamento
