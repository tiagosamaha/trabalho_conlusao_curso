\begin{center}
\textbf{RESUMO}
\end{center}
\singlespacing

\noindent O objetivo deste trabalho é apresentar o conceito de documentação executável como uma evolução da documentação tradicional, trazendo como vantagens adicionais a possibilidade do treinamento dos usuários e a validação das funcionalidades pelo cliente. Adicionalmente, este formato de documentação tende a se manter atualizado, já que além de documentar, assume funcionalidades de treinamento, agregando mais valor ao usuário. Assim, foi feito um estudo de caso na Biblioteca Digital da EPCT, estudo este que teve como resultados diretos a criação de um módulo de documentação executável para a referida aplicação, além de descrever o emprego das bibliotecas \textit{Amberjack2} e jQuery na implementação de documentação executável. \\

\noindent PALAVRAS-CHAVE: documentação executável, testes automatizados, treinamento
