\chapter{FUNDAMENTAÇÃO TEÓRICA}

Neste capítulo e apresentada uma revisão de assuntos relacionados na literatura, visando embasamento conceitual, para o entendimento deste trabalho.

\section{Origem dos testes na indústria}

Desde do início da indústria percebeu-se que os produtos precisam estar adequados a especificação. Produtos que são terminados com inconformidades tornam a produção mais lenta, gerando desperdícios como adaptação, retrabalho e refugo. O mesmo ocorre no processo de desenvolvimento de software, onde cada requisito deve atender as especificações solicitadas pelo cliente, caso contrário poderá ter funcionalidades inadequadas ou desnecessárias, que levam aos mesmos problemas da indústria.

A solução utilizada no secúlo XX, foi a inspeção ao final da produção. A indústria ocidental evoluiu ao fazer inspeções proativas (no estoque) e estatísticas por período de tempo. Entretanto o tempo entre a geração do erro e sua identificação diminui a capacidade de identificação da causa raíz dos problemas e de sua solução \cite{CARVALHO}.

Segundo \citeonline{HOLWEG}, em 1918 Sakichi Toyoda implantou em sua indústria de tecelagem uma máquina que verifica a linha de produção. Ao detectar anormalidades, como por exemplo a falta ou quebra da linha, a produção era parada e os operadores avisados, evitando assim a necessidade de operários para monitoramento da produção, e que o produto não atendesse a especificação. Desta maneira o tempo de identifição do problema é curto, evita vários problemas no processo de produção e aumenta as chances de análise e identificação das causas dos problemas.

Entretanto, todo o conhecimento gerado na industria oriental passou desapercebido, sem publicações científicas, até a década de 90, quando se iniciaram pesquisas a este respeito. Sendo assim, até os dias de hoje, percebe-se grandes falhas em empreendimentos nas diversas formas de indústria, inclusive na indústria de software.

\section{Prejuízos na indústria de software}

Software são desenvolvidos a mais de cinquenta anos e ainda tem grandes problemas com a qualidade e garantia de entrega. Segundo \citeonline{CHARETTE}, bilhões de dólares são gastos em projetos  que não são terminados, e 5\% a 15\% dos projetos iniciados serão abandonados antes de serem entregues ou considerados totalmente inadequados logo depois de seu término.

Desenvolver sistemas é caro, e o processo frequentemente sofre dificuldades com as metodologias adotadas, muitas delas não se adequam as atuais exigências do mercado. Por exemplo, o gerente do projeto e os desenvolvedores, devem estar sempre preparados para avaliar as demandas passadas pelos seus clientes (stakeholders) durante o processo de desenvolvimento do sistema. Na etapa de codificação, surgem constantes solicitações de modificações dos requisitos, os quais podem afetar o custo e a qualidade do software \cite{CERPA}.

O maior problema no processo de desenvolvimento é a existência de falhas no sistema, principalmente as detectáveis e previsíveis. Porém, muitas organizações ainda não consideram que a prevenção e detecção de falhas através de testes seja importante, mesmo que possam correr o risco de causar prejuízos ao cliente \cite{CHARETTE}. A tabela \ref{falhas_em_projetos} exemplifica tipos de falha em projetos de desenvolvimento de software é apresentada abaixo.

\begin{table}[ht]
	\centering
	\fontsize{8}{0}
	\caption{Percentual de projetos falhos por fator de falha}
	\label{falhas_em_projetos}
\begin{tabular}{lccc}

\hline

\textbf{Fatores de falha em projetos de software} & \multicolumn{3}{c}{\textbf{Porcentagem de projetos (\%)}}\tabularnewline

\cline{2-4}

& \textbf{In-House} & \textbf{Outsourced} & \textbf{Geral}\tabularnewline

\hline

Data de entrega influenciou no processo & 93,9 & 90,5 & 92,9\tabularnewline

\hline

Projeto estimado por baixo & 83,7 & 76,2 & 81,4\tabularnewline

\hline

Riscos não fora reavaliados, controlados ou gerenciados & 73,4 & 80,9 & 75,7\tabularnewline

\hline

A gerência não foi recompensada por longas horas de trabalho & 81,6 & 57,1 & 74,3\tabularnewline

\hline

Tomada de decisão foi feita sem informações necessárias & 83,7 & 47,6 & 72,9\tabularnewline

\hline

A gerência teve experiência desconfortável & 83,7 & 47,6 & 72,9\tabularnewline

\hline

Clientes não envolvidos na preparação do cronograma & 69,4 & 76,2 & 71,4\tabularnewline

\hline

Risco não está incorporado no planejamento do projeto & 65,3 & 80,9 & 70,0\tabularnewline

\hline

Controle de mudanças não monitorado/negociado efetivamente & 63,3 & 85,7 & 70,0\tabularnewline

\hline

Clientes e usuários tiveram espectativas não realísticas & 69,4 & 66,7 & 68,6\tabularnewline

\hline

Processo não teve revisões ao final de cada fase & 75,5 & 47,6 & 67,1\tabularnewline

\hline

Metodologia não apropriada para o projeto & 71,4 & 52,4 & 65,7\tabularnewline

\hline

Cronograma agressivo afetou a motivação da equipe & 69,4 & 57,1 & 65,7\tabularnewline

\hline

Mudanças no escopo durante o projeto & 67,3 & 57,1 & 64,3\tabularnewline

\hline

Cronograma teve efeito negativo na vida dos elementos da equipe & 71,4 & 42,9 & 62,9\tabularnewline

\hline

Projeto com equipe inadequada para cumprir o cronograma & 63,3 & 57,1 & 61,4\tabularnewline

\hline

Equipe adicionada tardiamente para cumprir um cronograma & 61,2 & 61,9 & 61,4\tabularnewline

\hline

Clientes não dão tempo suficiente para levantar requisitos & 61,2 & 57,1 & 60,0\tabularnewline

\hline

\end{tabular}
\end{table}

\section{Filosofia de Teste de Software}

Teste de software é uma das mais importantes atividades na garantia de qualidade. Segundo \citeonline{PRESSMAN}, a atividade de teste em um projeto de software pode consumir 40\% do esforço total do mesmo e chegar a custar de três a cinco vezes mais que todas as outras etapas da engenharia de software somadas.

A definição comumente aceita da atividade de teste de software é a de que "teste de software consiste de uma série de procedimentos pré-definidos para garantir que o código faça o que ele foi projetado para fazer e não tenha nenhum comportamento não desejado". Porém, \citeonline{MYERS}, aponta para o fato de que o objetivo primário de um teste de software é encontrar erros no código e não provar que o programa atinge aos objetivos: "Teste de software é o processo de se executar um programa com o propósito de descobrir erros". 

Muitos programadores assumem uma definição menos eficiente sobre a atividade de teste:

\begin{itemize}
\item A atividade de teste é o processo de demonstrar a inexistência de erros no software
\end{itemize}

Esse pensamento comum é inapropriado pois apesar da atividade de teste poder descobrir defeitos no software, ela é incapaz de mostrar a inexistência destes. No mundo real, é impraticável criar casos de teste para todas as combinações e cenários possíveis para aplicações mais complexas.

\begin{itemize}
\item O propósito da atividade de teste é mostrar que um programa executa suas funções corretamente.
\end{itemize}

Esta definição pode ser considerada incompleta pois mesmo que um programa atinja os objetivos para o qual foi projetado, ainda pode conter erros caso apresente funcionalidades adicionais.

\begin{itemize}
\item Teste de software é o processo de garantir que um programa faz o que foi projetado para fazer.
\end{itemize}

Apesar de uma atividade de teste bem executada poder demonstrar a conformidade entre os requisitos do cliente e as funcionalidades implementadas no software, este é um benefício secundário. Como qualquer outra atividade no processo de desenvolvimento, testes também devem agregar valor ao software, e portanto é importante encontrar defeitos. Neste sentido, a qualidade da atividade de teste será diretamente proporcial ao esforço em encontrar defeitos.

Por tudo isso, percebe-se que os testes de software cumprem uma função ética no processo de desenvolvimento, evitando prejuízos para as organizações ou pessoas. Um exemplo claro disso, é a responsabilidade atribuída ao desenvolvedor do algoritmo presente em um marca-passo, onde qualquer falha pode causar graves prejuízos ao usuário.

Portanto, para conduzir os testes efetivamente, existe toda uma complexidade advinda da variedade de algorítmos que precisarão ser testados. Várias técnicas propostas para endereçar os testes de algorítmos são apresentadas nos tópico a seguir.