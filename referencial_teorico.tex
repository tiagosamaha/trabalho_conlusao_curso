\chapter{FUNDAMENTAÇÃO TEÓRICA}

Neste capítulo e apresentada uma revisão de assuntos relacionados na literatura, visando embasamento conceitual, para o entendimento deste trabalho.

\section{Origem dos testes na indústria}

Desde do início da indústria percebeu-se que os produtos precisam estar adequados a especificação, sem isso a produção é mais lenta e gera desperdícios, como adaptação, retrabalho e refugo. (colocar comparacao a dev de software)

No inicio da indústria se inspecionava o final da produção, a indústria ocidental evoluiu para fazer inspeções pro-ativas (no estoque) e estatísticas por período de tempo. Entretanto o tempo entre a geração do erro e sua identificação diminui a capacidade de identificação do problema e de sua solução.

Máquina de tear e identificação e  prevenção de falhas.

\section{Prejuízos na indústria de software}

Software são desenvolvidos a mais de cinquenta anos e ainda tem grandes problemas com a qualidade e garantia de entrega. Segundo \citeonline{CHARETTE}, bilhões de dólares são gastos em projetos  que não são terminados, e que 5\% a 15\% dos projetos iniciados serão abandonados antes de serem entregues ou considerados totalmente inadequados logo depois de seu término.

Desenvolver sistemas é caro, e o processo frequentemente sofre dificuldades com as metodologias adotadas, muitas delas não se adequam as atuais exigências do mercado. Por exemplo, o gerente do projeto e os desenvolvedores, devem estar sempre preparados para avaliar as demandas passadas pelos seus clientes (stakeholders) durante o processo de desenvolvimento do sistema. Na etapa de codificação, surgem constantes solicitações de modificações dos requisitos, os quais podem afetar o custo e a qualidade do software \cite{CERPA}.

O maior problema no processo de desenvolvimento é a existência de falhas no sistema, principalmente as detectáveis e previsíveis. Porém, muitas organizações ainda não consideram que a prevenção e detecção de falhas através de testes seja importante, mesmo que possam correr o risco de causar prejuízos ao cliente \cite{CHARETTE}.